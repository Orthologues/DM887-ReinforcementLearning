\documentclass[a4paper,12pt,oneside]{article}
\usepackage[utf8]{inputenc}
\usepackage{graphicx}
\usepackage{amsmath,amssymb}
\usepackage{amsthm}
\DeclareMathOperator{\argmax}{arg,max}
\DeclareMathOperator{\argmin}{arg,min}
\setlength\parindent{0pt}
\setlength{\topmargin}{-2cm}
\setlength{\textheight}{24cm}

\hoffset= -70pt
\textwidth=540pt
\numberwithin{equation}{section}

\usepackage{algpseudocode}
\usepackage{algorithm}
\usepackage{xcolor}
\usepackage{lmodern} % Load a font package


\begin{document}

\title{DM887 Assignment2 Q1}
\author{Jiawei Zhao}
\date{19.03.2024}
\maketitle
\newcommand{\mycomment}[1]{{\fontfamily{lmss}\selectfont\textcolor{blue}{[#1]}}} % Sans-serif and blue
    \begin{algorithm}
    \caption {Least-Squares Temporal Differences (LSTD) Deep Q-Learning with Nonlinear Feature Extraction that maps a state to lower-dimensional latent embedding. While the action-value function should be a linear function of the output of the feature extractor.}
        \begin{algorithmic}[1]
        \State Initialize a variational autoencoder $\mathbf{A}$ as the feature extraction network with initial weights $\theta_0$ 
        \mycomment {$\theta_0$ are drawn from a zero-mean Gaussian distribution}
        \State Initialize the number of maximum episode \(N_0\) = 10, \(N_1\) = 10, and \(N_2\) = 30 for each phase of training
        \mycomment {It would be a wise choice to seperate the warm-up phase, the antoencoder improvement phase, and the LSTD improvement phase}
        \mycomment {Initialization of training hyperparameters}
        \State Initialize the number of maximum time step per episode \(t\) = 5000
        \State Initialize the weights \(w_0\) for linear approximation of the action-value function \(y = f(phi(s))\) which will be used for LSTD
        \State Initialize a replay memory buffer $\mathcal{D}$ with a capacity \(N = t (N_0 + N_1 + N_2)\)
        \State Initialize a discount factor $\gamma=0.9$ 
        \State Initialize the minibatch size of DQL as \(k=32\) 
        \mycomment{A relatively large discount factor encourages long-term planning and faster convergence during training}
        \clearpage
        \mycomment {The warm-up phase starts here}
        \State Freeze the weights of $\mathbf{A}$ as $\theta_0$
        \State Freeze the weights of the \(f(phi(s))\) as \(w_0\)
        \For {\(episode\) \(e = 1\) to \(N_0\)}
            \State Initialize state \(s_0\) 
            \State Preprocess \(s\) into \(s_1\) to adapt it as input of $\mathbf{A}$
            \mycomment{preprocessing of high-dimensional states is necessary w.r.t. autoenconders}
            \For {each time step \(t\) = 1 to \(T\)}
                \State Encode state \(s_t\) using $\mathbf{A}$ to get latent embedding $\phi(s_t)$
                \State Select action $a_t$ using $\epsilon$-greedy policy with $\epsilon=0.3$
                \mycomment{The learning curves of Game B of Assignment 1 corroborate that a relatively high $\epsilon$ could improve \(Q\) more efficiently when an \(e\) does not end prematurely in the majority of cases}
                \State Execute $a_t$ and obtain reward $r_t$ and new state $s_{t+1}$
                \State Store the transition of the current \(t\), i.e. $(s_t, a_t, r_t, s_{t+1})$ in $\mathcal{D}$
                \State Encode state \(s_t\) using $\mathbf{A}$ to get latent embedding $\phi(s_t)$
                \State Select action $a_t$ using $\epsilon$-greedy policy with $\epsilon=0.3$
                \mycomment{The learning curves of Game B of Assignment 1 corroborate that a relatively high $\epsilon$ could improve \(Q\) more efficiently when an \(e\) does not end prematurely in the majority of cases}
                \State Execute $a_t$ and obtain reward $r_t$ and new state $s_{t+1}$
                \State Store the transition of the current \(t\), i.e. $(s_t, a_t, r_t, s_{t+1})$ in $\mathcal{D}$
            \EndFor
        \EndFor
        \end{algorithmic}[1]
    \end{algorithm}
    
    \clearpage
    \begin{algorithm}
    \caption{The LSTD update starts here}
        \begin{algorithmic}[1]
        \State Freeze the weights of $\mathbf{A}$ as $\theta_0$
        \State Unfreeze the weights of \(f(phi(s))\), i.e. \(w\)
        \For {\(episode\) \(e = 1\) to \(N_1\)}
            \State Repeat the same steps at phase 1
            \For {each time step \(t\) = 1 to \(T\)}
                \State Repeat the same steps at phase 1
                \\
                \mycomment{Start using the LSTD algorithm to update \(w\)}
                \State Sample a minibatch of transitions from $\mathcal{D}$ with a batchsize with a randomly selected  \

            \EndFor
        \EndFor
        \end{algorithmic}[1]
    \end{algorithm}

    \clearpage
    \begin{algorithm}
    \caption{The autoencoder + LSTD training starts here}
        \begin{algorithmic}[1]
        \State Unfreeze the weights of $\mathbf{A}$, i.e. $\theta$
        \For {\(episode\) \(e = 1\) to \(N_2\)}
            \State Repeat the same steps at phase 1
            \For {each time step \(t\) = 1 to \(T\)}
                \State Repeat the same steps at phase 1
                \\
                \mycomment{Start using the online LSTD algorithm to update \(w\)}
                \State Sample a minibatch of transitions from $\mathcal{D}$ with a batchsize \
            \EndFor
        \EndFor
        \end{algorithmic}[1]    
    \end{algorithm}


\end{document}